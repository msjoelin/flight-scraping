\documentclass[]{article}
\usepackage{lmodern}
\usepackage{amssymb,amsmath}
\usepackage{ifxetex,ifluatex}
\usepackage{fixltx2e} % provides \textsubscript
\ifnum 0\ifxetex 1\fi\ifluatex 1\fi=0 % if pdftex
  \usepackage[T1]{fontenc}
  \usepackage[utf8]{inputenc}
\else % if luatex or xelatex
  \ifxetex
    \usepackage{mathspec}
  \else
    \usepackage{fontspec}
  \fi
  \defaultfontfeatures{Ligatures=TeX,Scale=MatchLowercase}
\fi
% use upquote if available, for straight quotes in verbatim environments
\IfFileExists{upquote.sty}{\usepackage{upquote}}{}
% use microtype if available
\IfFileExists{microtype.sty}{%
\usepackage{microtype}
\UseMicrotypeSet[protrusion]{basicmath} % disable protrusion for tt fonts
}{}
\usepackage[margin=1in]{geometry}
\usepackage{hyperref}
\hypersetup{unicode=true,
            pdftitle={Flight Price Web Scraping},
            pdfauthor={Marcus},
            pdfborder={0 0 0},
            breaklinks=true}
\urlstyle{same}  % don't use monospace font for urls
\usepackage{color}
\usepackage{fancyvrb}
\newcommand{\VerbBar}{|}
\newcommand{\VERB}{\Verb[commandchars=\\\{\}]}
\DefineVerbatimEnvironment{Highlighting}{Verbatim}{commandchars=\\\{\}}
% Add ',fontsize=\small' for more characters per line
\usepackage{framed}
\definecolor{shadecolor}{RGB}{248,248,248}
\newenvironment{Shaded}{\begin{snugshade}}{\end{snugshade}}
\newcommand{\KeywordTok}[1]{\textcolor[rgb]{0.13,0.29,0.53}{\textbf{#1}}}
\newcommand{\DataTypeTok}[1]{\textcolor[rgb]{0.13,0.29,0.53}{#1}}
\newcommand{\DecValTok}[1]{\textcolor[rgb]{0.00,0.00,0.81}{#1}}
\newcommand{\BaseNTok}[1]{\textcolor[rgb]{0.00,0.00,0.81}{#1}}
\newcommand{\FloatTok}[1]{\textcolor[rgb]{0.00,0.00,0.81}{#1}}
\newcommand{\ConstantTok}[1]{\textcolor[rgb]{0.00,0.00,0.00}{#1}}
\newcommand{\CharTok}[1]{\textcolor[rgb]{0.31,0.60,0.02}{#1}}
\newcommand{\SpecialCharTok}[1]{\textcolor[rgb]{0.00,0.00,0.00}{#1}}
\newcommand{\StringTok}[1]{\textcolor[rgb]{0.31,0.60,0.02}{#1}}
\newcommand{\VerbatimStringTok}[1]{\textcolor[rgb]{0.31,0.60,0.02}{#1}}
\newcommand{\SpecialStringTok}[1]{\textcolor[rgb]{0.31,0.60,0.02}{#1}}
\newcommand{\ImportTok}[1]{#1}
\newcommand{\CommentTok}[1]{\textcolor[rgb]{0.56,0.35,0.01}{\textit{#1}}}
\newcommand{\DocumentationTok}[1]{\textcolor[rgb]{0.56,0.35,0.01}{\textbf{\textit{#1}}}}
\newcommand{\AnnotationTok}[1]{\textcolor[rgb]{0.56,0.35,0.01}{\textbf{\textit{#1}}}}
\newcommand{\CommentVarTok}[1]{\textcolor[rgb]{0.56,0.35,0.01}{\textbf{\textit{#1}}}}
\newcommand{\OtherTok}[1]{\textcolor[rgb]{0.56,0.35,0.01}{#1}}
\newcommand{\FunctionTok}[1]{\textcolor[rgb]{0.00,0.00,0.00}{#1}}
\newcommand{\VariableTok}[1]{\textcolor[rgb]{0.00,0.00,0.00}{#1}}
\newcommand{\ControlFlowTok}[1]{\textcolor[rgb]{0.13,0.29,0.53}{\textbf{#1}}}
\newcommand{\OperatorTok}[1]{\textcolor[rgb]{0.81,0.36,0.00}{\textbf{#1}}}
\newcommand{\BuiltInTok}[1]{#1}
\newcommand{\ExtensionTok}[1]{#1}
\newcommand{\PreprocessorTok}[1]{\textcolor[rgb]{0.56,0.35,0.01}{\textit{#1}}}
\newcommand{\AttributeTok}[1]{\textcolor[rgb]{0.77,0.63,0.00}{#1}}
\newcommand{\RegionMarkerTok}[1]{#1}
\newcommand{\InformationTok}[1]{\textcolor[rgb]{0.56,0.35,0.01}{\textbf{\textit{#1}}}}
\newcommand{\WarningTok}[1]{\textcolor[rgb]{0.56,0.35,0.01}{\textbf{\textit{#1}}}}
\newcommand{\AlertTok}[1]{\textcolor[rgb]{0.94,0.16,0.16}{#1}}
\newcommand{\ErrorTok}[1]{\textcolor[rgb]{0.64,0.00,0.00}{\textbf{#1}}}
\newcommand{\NormalTok}[1]{#1}
\usepackage{graphicx,grffile}
\makeatletter
\def\maxwidth{\ifdim\Gin@nat@width>\linewidth\linewidth\else\Gin@nat@width\fi}
\def\maxheight{\ifdim\Gin@nat@height>\textheight\textheight\else\Gin@nat@height\fi}
\makeatother
% Scale images if necessary, so that they will not overflow the page
% margins by default, and it is still possible to overwrite the defaults
% using explicit options in \includegraphics[width, height, ...]{}
\setkeys{Gin}{width=\maxwidth,height=\maxheight,keepaspectratio}
\IfFileExists{parskip.sty}{%
\usepackage{parskip}
}{% else
\setlength{\parindent}{0pt}
\setlength{\parskip}{6pt plus 2pt minus 1pt}
}
\setlength{\emergencystretch}{3em}  % prevent overfull lines
\providecommand{\tightlist}{%
  \setlength{\itemsep}{0pt}\setlength{\parskip}{0pt}}
\setcounter{secnumdepth}{0}
% Redefines (sub)paragraphs to behave more like sections
\ifx\paragraph\undefined\else
\let\oldparagraph\paragraph
\renewcommand{\paragraph}[1]{\oldparagraph{#1}\mbox{}}
\fi
\ifx\subparagraph\undefined\else
\let\oldsubparagraph\subparagraph
\renewcommand{\subparagraph}[1]{\oldsubparagraph{#1}\mbox{}}
\fi

%%% Use protect on footnotes to avoid problems with footnotes in titles
\let\rmarkdownfootnote\footnote%
\def\footnote{\protect\rmarkdownfootnote}

%%% Change title format to be more compact
\usepackage{titling}

% Create subtitle command for use in maketitle
\newcommand{\subtitle}[1]{
  \posttitle{
    \begin{center}\large#1\end{center}
    }
}

\setlength{\droptitle}{-2em}

  \title{Flight Price Web Scraping}
    \pretitle{\vspace{\droptitle}\centering\huge}
  \posttitle{\par}
    \author{Marcus}
    \preauthor{\centering\large\emph}
  \postauthor{\par}
      \predate{\centering\large\emph}
  \postdate{\par}
    \date{February 14, 2019}


\begin{document}
\maketitle

\subsection{Flight Web Scraping}\label{flight-web-scraping}

I'm traveling very frequently, and this includes searching for flights.
Price is of course one important factor when deciding in the decision
process, and I guess that most of us have noticed the fluctuations in
flight prices. It happened that I searched for a flight, the price
looked good, but didnt book the flight. The next day, the price had
doubled (at least it felt like that).

To take a closer look at this without having to search for flights every
day, I wrote an R-script to scrape the data. As some sites are making it
hard to scrape the data (with recaptcha or homepages where the elements
arent that hard to scrape), the first step was to identify a flight
search engines where the scraping could be performed without too much of
effort. Expedia turned out to be an ``easy'' site to scrape.

\begin{Shaded}
\begin{Highlighting}[]
\KeywordTok{library}\NormalTok{(readr)      }\CommentTok{# Read data in data }
\KeywordTok{library}\NormalTok{(lubridate)  }\CommentTok{# Date handling}
\KeywordTok{library}\NormalTok{(stringr)    }\CommentTok{# Text handling}

\KeywordTok{library}\NormalTok{(tidyr)      }\CommentTok{# Data cleaning}
\KeywordTok{library}\NormalTok{(dplyr)      }\CommentTok{# Data cleaning}

\KeywordTok{library}\NormalTok{(ggplot2)    }\CommentTok{# Visualization}

\NormalTok{df <-}\StringTok{ }\KeywordTok{read.csv}\NormalTok{(}\StringTok{"flightdata.csv"}\NormalTok{, }\DataTypeTok{stringsAsFactors =} \OtherTok{TRUE}\NormalTok{)}

\KeywordTok{head}\NormalTok{(df)}
\end{Highlighting}
\end{Shaded}

\begin{verbatim}
##   Journey Origin Destination DepartureDate ReturnDate
## 1 BER-BKK    BER         BKK    2019-01-20 2019-01-24
## 2 BER-BKK    BER         BKK    2019-01-20 2019-01-25
## 3 BER-BKK    BER         BKK    2019-01-20 2019-01-26
## 4 BER-BKK    BER         BKK    2019-01-20 2019-01-27
## 5 BER-BKK    BER         BKK    2019-01-21 2019-01-25
## 6 BER-BKK    BER         BKK    2019-01-21 2019-01-26
##             TravelDates ScrapeDate price price_stand DaysBeforeDeparture
## 1 2019-01-20-2019-01-24 2019-01-14  5384   0.5625268                   6
## 2 2019-01-20-2019-01-25 2019-01-14  5153   0.4812452                   6
## 3 2019-01-20-2019-01-26 2019-01-14  5384   0.5625268                   6
## 4 2019-01-20-2019-01-27 2019-01-14  5485   0.5980655                   6
## 5 2019-01-21-2019-01-25 2019-01-14  5858   0.7293124                   7
## 6 2019-01-21-2019-01-26 2019-01-14  5858   0.7293124                   7
\end{verbatim}

First I want to check wihc date the scraping was performed.

\begin{Shaded}
\begin{Highlighting}[]
\KeywordTok{ggplot}\NormalTok{(df, }\KeywordTok{aes}\NormalTok{(}\DataTypeTok{x=}\NormalTok{ScrapeDate))}\OperatorTok{+}
\StringTok{  }\KeywordTok{stat_count}\NormalTok{(}\DataTypeTok{geom=}\StringTok{"bar"}\NormalTok{) }\OperatorTok{+}\StringTok{ }
\StringTok{  }\KeywordTok{coord_flip}\NormalTok{()}
\end{Highlighting}
\end{Shaded}

\includegraphics{flight_price_web_scraping_files/figure-latex/unnamed-chunk-2-1.pdf}

\begin{Shaded}
\begin{Highlighting}[]
\NormalTok{df <-}\StringTok{ }\KeywordTok{read.csv}\NormalTok{(}\StringTok{"flightdata.csv"}\NormalTok{, }\DataTypeTok{stringsAsFactors =} \OtherTok{TRUE}\NormalTok{)}

\KeywordTok{head}\NormalTok{(df)}
\end{Highlighting}
\end{Shaded}

\begin{verbatim}
##   Journey Origin Destination DepartureDate ReturnDate
## 1 BER-BKK    BER         BKK    2019-01-20 2019-01-24
## 2 BER-BKK    BER         BKK    2019-01-20 2019-01-25
## 3 BER-BKK    BER         BKK    2019-01-20 2019-01-26
## 4 BER-BKK    BER         BKK    2019-01-20 2019-01-27
## 5 BER-BKK    BER         BKK    2019-01-21 2019-01-25
## 6 BER-BKK    BER         BKK    2019-01-21 2019-01-26
##             TravelDates ScrapeDate price price_stand DaysBeforeDeparture
## 1 2019-01-20-2019-01-24 2019-01-14  5384   0.5625268                   6
## 2 2019-01-20-2019-01-25 2019-01-14  5153   0.4812452                   6
## 3 2019-01-20-2019-01-26 2019-01-14  5384   0.5625268                   6
## 4 2019-01-20-2019-01-27 2019-01-14  5485   0.5980655                   6
## 5 2019-01-21-2019-01-25 2019-01-14  5858   0.7293124                   7
## 6 2019-01-21-2019-01-26 2019-01-14  5858   0.7293124                   7
\end{verbatim}

As it seems, some days the scraping wasnt done fully.

\includegraphics{flight_price_web_scraping_files/figure-latex/unnamed-chunk-3-1.pdf}

\begin{verbatim}
## <ggproto object: Class CoordFlip, CoordCartesian, Coord, gg>
##     aspect: function
##     backtransform_range: function
##     clip: on
##     default: FALSE
##     distance: function
##     expand: TRUE
##     is_free: function
##     is_linear: function
##     labels: function
##     limits: list
##     modify_scales: function
##     range: function
##     render_axis_h: function
##     render_axis_v: function
##     render_bg: function
##     render_fg: function
##     setup_data: function
##     setup_layout: function
##     setup_panel_params: function
##     setup_params: function
##     transform: function
##     super:  <ggproto object: Class CoordFlip, CoordCartesian, Coord, gg>
\end{verbatim}

\begin{verbatim}
##   Journey Origin Destination DepartureDate ReturnDate
## 1 BER-BKK    BER         BKK    2019-01-20 2019-01-24
## 2 BER-BKK    BER         BKK    2019-01-20 2019-01-25
## 3 BER-BKK    BER         BKK    2019-01-20 2019-01-26
## 4 BER-BKK    BER         BKK    2019-01-20 2019-01-27
## 5 BER-BKK    BER         BKK    2019-01-21 2019-01-25
## 6 BER-BKK    BER         BKK    2019-01-21 2019-01-26
##             TravelDates ScrapeDate price price_stand DaysBeforeDeparture
## 1 2019-01-20-2019-01-24 2019-01-14  5384   0.5625268                   6
## 2 2019-01-20-2019-01-25 2019-01-14  5153   0.4812452                   6
## 3 2019-01-20-2019-01-26 2019-01-14  5384   0.5625268                   6
## 4 2019-01-20-2019-01-27 2019-01-14  5485   0.5980655                   6
## 5 2019-01-21-2019-01-25 2019-01-14  5858   0.7293124                   7
## 6 2019-01-21-2019-01-26 2019-01-14  5858   0.7293124                   7
\end{verbatim}

\subsection{Including Plots}\label{including-plots}

You can also embed plots, for example:

\includegraphics{flight_price_web_scraping_files/figure-latex/pressure-1.pdf}

Note that the \texttt{echo\ =\ FALSE} parameter was added to the code
chunk to prevent printing of the R code that generated the plot.


\end{document}
